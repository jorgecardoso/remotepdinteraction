%!TEX root = mieic.tex
\chapter{Testes} \label{chap:testes}

\section*{}

Os testes realizados são uma fase importante no desenvolvimento de qualquer produto, permitindo descobrir alguma falhas no trabalho efetuado e dando sugestões de melhorias futuras.
Neste capítulo são descritos os diversos tipos de testes bem como as conclusões a que os mesmos permitiram chegar.

\section{Testes Iterativos}

	Ao longo de todo o desenvolvimento, sempre que uma nova funcionalidade era implementada surgia a necessidade de testar de forma a perceber se esta efetuava a ação desejada.

	Este tipo de testes, permite a quem está a desenvolver a aplicação, obter de forma rápida \textit{feedback} do trabalho acabado de realizar, prevenindo erros futuros semelhantes e trabalho desnecessário.

	Por exemplo, na solução implementada, aquando da criação dos tipos de controlo, primeiro foi definido e implementado apenas um, neste caso o \textit{widget joystick}, que foi testado verificando se a comunicação com a aplicação existia, e se as setas executavam no jogo as ações supostas. Assim qualquer erro detetado e corrigido contribuiu para uma maior facilidade na criação dos restantes \textit{widgets}, diminuindo a correção de erros no final da implementação.


\section{Testes Finais}
	
	Os testes realizados para verificar se as funcionalidades implementadas correspondem ao especificado inicialmente são de fácil realização e apenas requerem que alguém experimente o que está desenvolvido. Contudo neste projeto era também importante perceber de que modo outros programadores poderiam usar a \textit{framework} desenvolvida com o intuito de definirem alguns tipos de controlo para as sua próprias aplicações.

	Para a execução destes testes foram escolhidos três estudantes do 5º ano do Mestrado Integrado em Engenharia Informática e Computação. Ao longo do seu percurso académico tiveram duas unidades curriculares onde fizeram \textit{web development} e tiveram de interagir com HTML5, CSS3 e JavaScript.

	Um dos elementos, sujeito A, apenas possuía a experiência adquirida durante o percurso académico. Os outros dois elementos, sujeito B e C, já tinham realizado alguns projetos extra-curriculares que os obrigou a explorar um pouco mais estas tecnologias, fornecendo-lhes uma maior experiência e conhecimento. 

	Inicialmente foram explicados aos intervenientes os objetivos do projeto, para que eles estivessem enquadrados e pudessem perceber os seu papel nesta atividade. Também lhes foi dada, antes de iniciarem, uma breve explicação sobre o que teriam de fazer para usar a \textit{framework}.

	De seguida foi-lhes dado todo o projeto desenvolvido, com a exceção de dois ficheiros. Um que continha o código do jogo implementado e o outro que dizia respeito à implementação dos controlos pretendidos.

	O objetivo era que desenvolvessem uma pequena aplicação e que usassem para a definição dos controlos exigidos a \textit{API} desenvolvida. Os participantes dispunham de 2 horas e 30 minutos para a realização desta atividade. O tempo dado foi utilizado de maneiras diferentes pelos participantes. Enquanto o utilizador A apenas se preocupou em implementar uma solução, os outros dois tiveram a preocupação de fazer perguntas pertinentes sobre a \textit{framework} e sugerir alterações.

	Das pessoas acima referidas, todas optaram pela implementação de jogos, contudo escolheram exemplos diferentes. Na tabela ~\ref{table:Resultados}, é possível ver as aplicações escolhidas, bem como um pequeno resumo das soluções obtidas.

	\begin{table}[ht]
	\centering

	\begin{tabular}{ |l|l|l|l| }
	\hline
	\textbf{Elemento} & \textbf{Aplicação} & \textbf{Funcional} & \textbf{Multi-Utilizador} \\ 
	\hline
	\textbf{A} & Jogo da Forca & Sim & Não \\
	\hline
	\textbf{B} & Corrida de automóveis & Sim & Sim \\
	\hline
	\textbf{C} & Tetris & Sim & Não \\
	\hline
	\end{tabular}

	\caption{Resultados Obtidos}
	\label{table:Resultados}
	\end{table}

	Em todas os exemplos, os estudantes conseguiram com sucesso implementar os tipos de controlo existentes e ter uma aplicação funcional, contudo em todos eles foram referidas pequenas falhas ou sugestões relativas ao projeto.

	O decorrer dos testes permitiu obter diferentes opiniões de acordo com o desenvolver da solução:

		\begin{itemize}
		\item \textbf{Jogo da forca: } Só foi usado o controlo que exigia introdução de texto. O Utilizador inseria o seu nome de jogador e poderia começar a jogar.

		Só foi conseguida a versão para um jogador...

		\item \textbf{Corrida de automóveis: } Foi implementada com sucesso, sendo possível usar os três diferentes tipos de controlo, começando por introduzir o nome de jogador, seguido da escolha do \textit{widget} preferido para controlar o veículo.

		A solução apresentada suportava o modo multi-jogador, em que os jogadores eram distinguidos pela cor do automóvel e respetivo nome.

		Foi sugerida a criação de um novo tipo de controlo que usasse o acelerómetro do dispositivo, de modo a permitir uma interação através do movimento do mesmo.

		\item \textbf{Tetris: } À semelhança dos anteriores também este exemplo se encontrava funcional no fim do tempo dado. Foram usados dois diferentes tipos de controlo, a caixa de texto para a escolha do nome do utilizador e o \textit{joystick} para controlar as peças do jogo. 

		Tal como aconteceu no exemplo do jogo da forca, este apenas podia ser jogado por um utilizador. A pessoa que o implementou possuía a solução para permitir um modo multi-jogador, no entanto por falta de tempo não houve a possibilidade de implementar. A solução passaria pela existência de um número de ``tabuleiros'' de tetris igual ao número de jogadores.

		\end{itemize}

	Adicionalmente foram também sugeridas pequenas alterações a nível de funções da \textit{API} desenvolvida que foram corrigidas. O método \textit{setOptions(options)}, que permite ao programador alterar a informação que é enviada através do \textit{widget}, inicialmente estava definida para cada \textit{widget} independentemente. Após a realização dos testes foi sugerido pelo sujeito B, que faria sentido ser um método geral, uma vez que a informação enviada, independentemente do tipo de controlo serão iguais para determinada aplicação.
	O sujeito C sugeriu que seria preferível o programador não ter de seguir um estrutura específica para o ficheiro \textit{HTML}, pois condiciona a liberdade do mesmo. 

	Ainda durante a realização destes testes foi possível constatar que existem algumas alterações a efetuar no código original para a implementação de novos aplicações, o que dificulta o trabalho de futuras pessoas que não estejam familiarizadas com código fornecido. Por exemplo, é necessário alterar três vezes o endereço onde a aplicação está a correr, o que deveria ser evitado para que futuros utilizadores não necessitem de alterar ficheiros exteriores à \textit{framework}. É também necessário que o programador preste especial atenção ao ficheiro \textit{index.js}, local onde são chamadas funções específicas da aplicação implementada. 

\section{Testes Públicos}

	Aquando da definição e especificação do projeto foi referido que seria vantajoso testar a solução implementada em ecrãs públicos, no entanto devido a algum atraso no desenvolvimento, o mesmo não foi possível.

	A realização destes testes permitiria tirar algumas conclusões relativas à utilização por parte dos transeuntes, que poderiam ser resposta às seguintes questões:

	\begin{itemize}
	\item \textbf{Antes de efetuar a ligação com o ecrã:}
		\begin{itemize}
		\item Que riscos estou a correr se me ligar a este ecrã?
		\item O que tenho de fazer para me conseguir ligar?
		\end{itemize}
	\item \textbf{Após estar conectado com o ecrã}:
		\begin{itemize}
		\item O que tenho de fazer para usar a aplicação?
		\item Como sei quem sou eu durante a interação?
		\item O que faço quando desejar terminar?
		\end{itemize}
	\end{itemize} 

\section{Conclusões}

	Após a realização dos testes acima descritos, houve a possibilidade de perceber realmente de que modo a aplicação era vista por alguém que não estava a par do seu desenvolvimento e que alterações deveriam ser efetuadas para obter uma melhor \textit{performance}.

	A sua realização permitiu descobrir falhas que puderam ser corrigidas ainda antes de dar por concluída a solução final e outras que ficaram registadas como melhorias a efetuar futuramente caso exista a hipótese de dar continuidade ao trabalho realizado.

	A realização dos últimos testes referidos poderá acontecer em qualquer altura sendo apenas necessário a existência de infra-estruturas adequadas.



	

