%!TEX root = mieic.tex
\chapter{Testes Realizados} \label{chap:testes}

\section*{}

Os testes realizados são uma fase importante no desenvolvimento de qualquer produto, permitindo descobrir alguma falhas no trabalho realizado e dando sugestões de melhorias futuras.
Neste capítulo são descritos os testes realizados bem como as conclusões a que os mesmos permitiram chegar.

\subsection*{Testes Iterativos}

	Ao longo de todo o desenvolvimento, sempre que uma nova funcionalidade era implementada surgia a necessidade de testar de forma a perceber se esta efetuava a ação desejada.

	Este tipo de testes, permite a quem está a desenvolver a aplicação, obter de forma rápida \textit{feedback} do trabalho acabado de realizar, prevenindo erros futuros semelhantes e trabalho desnecessário.

	Por exemplo, na solução implementada, aquando da criação dos tipos de controlo, primeiro foi definido e implementado apenas um, neste caso o \textit{widget joystick}, que foi testado verificando se a comunicação com a aplicação existia, e se as setas executavam no jogo as ações supostas. Assim qualquer erro detetado e corrigido contribuiu para uma maior facilidade na criação dos restantes \textit{widgets}, diminuindo a correção de erros no final da implementação.

\subsection*{Testes Finais}
	
	Os testes realizados para verificar se as funcionalidades implementadas correspondem ao especificado inicialmente são de fácil realização e apenas requerem que alguém experimente o que está desenvolvido. Contudo neste projeto era também importante perceber de que modo outros programadores poderiam usar a \textit{framework} desenvolvida com o intuito de definirem alguns tipos de controlo para as sua próprias aplicações.

	Foi pedido a três estudantes do 5º ano do mestrado integrado em engenharia informática e computação, que desenvolvessem uma pequena aplicação e que usassem para a definição dos controlos exigidos a \textit{API} desenvolvida, para que fosse possível perceber as dificuldades encontradas e que alterações seriam vantajosas.   

\subsection*{Testes Quantitativos}

	

\subsection*{Testes Públicos}

	Aquando da definição e especificação do projeto foi referido que seria vantajoso testar a solução implementada em ecrãs públicos, no entanto devido a algum atraso no desenvolvimento, o mesmo não foi possível.

	A realização destes testes permitira tirar algumas conclusões relativas à utilização por parte dos transeuntes, que poderiam ser resposta às seguintes questões:

	\begin{itemize}
	\item \textbf{Antes de efetuar a ligação com o ecrã:}
		\begin{itemize}
		\item Que riscos estou a correr se me ligar a este ecrã?
		\item O que tenho de fazer para me conseguir ligar?
		\end{itemize}
	\item \textbf{Após estar conectado com o ecrã}
		\begin{itemize}
		\item O que tenho de fazer para usar a aplicação?
		\item Como sei quem sou eu durante a interação?
		\item O que faço quando desejar terminar?
		\end{itemize}
	\end{itemize} 

\subsection*{Conclusões}

	Após a realização dos testes acima descritos, houve a possibilidade de perceber realmente de que modo a aplicação era vista por alguém que não estava a par do seu desenvolvimento e que alterações deveriam ser efetuadas para obter uma melhor \textit{performance}.

	A sua realização permitiu descobrir falhas que puderam ser corrigidas ainda antes de dar por concluída a solução final e outras que ficaram registadas como melhorias a efetuar futuramente caso exista a hipótese de dar continuidade ao trabalho realizado.

	A realização dos últimos testes referidos poderá acontecer em qualquer altura sendo apenas necessário a existência de infra-estruturas adequadas.



	

