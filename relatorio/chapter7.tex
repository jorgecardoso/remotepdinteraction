%!TEX root = mieic.tex
\chapter{Conclusões e Trabalho Futuro} \label{chap:concl}

\section*{}


	
\subsection*{Trabalho Futuro}

Tendo em conta o trabalho realizado, no futuro, haverá facilidade em continuar o desenvolvimento da \textit{framework} apresentada, e ainda a introdução de novas aplicações, bem como novos métodos a partir dos quais, o utilizador, se pode ligar a um ecrã.

Uma vez que apenas estão definidos três diferentes tipos de controlo, será plausível a introdução de novos \textit{widgets}, como por exemplo, a criação de um que permita a seleção de objetos e consequente manipulação dos mesmos. *COMO??*

A continuidade das aplicações será da responsabilidade de alguém que deseje implementar um sistema de interação, apenas será necessário que a mesma se encontre desenvolvida em JavaScript.

Na solução apresentada a leitura do \textit{QR code} é a única forma disponibilizada para que o utilizador possa interagir com o jogo, será vantajoso, numa abordagem futura, a criação de novos métodos que facilitem e incentivem o uso por parte dos transeuntes.



