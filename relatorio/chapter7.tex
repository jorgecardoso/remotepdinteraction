%!TEX root = mieic.tex
\chapter{Conclusões e Trabalho Futuro} \label{chap:concl}

\section*{}

É cada vez mais comum em áreas públicas a existência de ecrãs públicos com os quais os transeuntes podem interagir usando diferentes tecnologias.

Na elaboração deste projeto era pretendida uma solução que disponibilizasse uma interação, multi-utilizador, baseada no paradigma de manipulação direta através do dispositivo móvel do cliente. Também era esperado desenvolver uma \textit{framework} que facilitasse o desenvolvimento de aplicações para esse conceito.

No início tinha sido planeada a implementação de diferentes tipos de aplicações que possibilitassem o uso de controlos diferentes para uma melhor avaliação da \textit{framework} desenvolvida, contudo com o avançar do tempo isso tornou-se impossível, existindo apenas um exemplo, o jogo da \textit{Snake}, no qual foram testados todos os \textit{widgets} criados. 

A \textit{framework} apresentada foi testada, permitindo alterações de modo a melhorar a solução final. No entanto ficaram por efetuar algumas sugestões dadas que seriam uma mais valia aquando da implementação de novas aplicações.

Em suma, apesar de nem tudo o que foi definido inicialmente ter sido conseguido, é possível afirmar que os objetivos gerais foram alcançados faltando apenas a implementação de mais exemplos específicos de modo a abranger uma maior leque de aplicações e consequentes problemáticas.

\subsection*{No Futuro}

Tendo em conta o trabalho realizado, no futuro, haverá facilidade em continuar o desenvolvimento da \textit{framework} apresentada, e ainda a introdução de novas aplicações, bem como novos métodos a partir dos quais, o utilizador, se pode ligar a um ecrã.

Uma vez que apenas estão definidos três diferentes tipos de controlo, será plausível a introdução de novos \textit{widgets}, como por exemplo, a criação de um que permita a seleção de objetos e consequente manipulação dos mesmos e a utilização do acelerómetro do dispositivo como forma controlo.

As sugestões dadas durante a realização dos testes, que não foram realizadas, serão um possível ponto de partida se for desejada a continuidade do desenvolvimento da API, facilitando todo o trabalho posterior. 

A continuidade das aplicações será da responsabilidade de alguém que deseje implementar um sistema de interação, apenas será necessário que a mesma se encontre desenvolvida em JavaScript.

Na solução apresentada a leitura do \textit{QR code} é a única forma disponibilizada para que o utilizador possa interagir com o jogo, será vantajoso, numa abordagem futura, a criação de novos métodos que facilitem e incentivem o uso por parte dos transeuntes.

Uma das problemáticas que talvez ficou mais aquém das expectativas, e que no futuro deveria ser melhorada, centra-se na distinção dos utilizadores, para a qual existem múltiplas soluções, e apenas uma foi implementada.

