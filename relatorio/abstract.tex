%!TEX root = mieic.tex
\chapter*{Resumo}

Os ecrãs públicos digitais têm cada vez mais uma forte presença no nosso dia-a-dia, contudo poucos são os que permitem uma manipulação direta por parte do utilizador. Atualmente, a sua função é maioritariamente publicitar determinado serviço ou produto, e a maior parte das vezes até passam despercebidos aos transeuntes.

É importante reconhecer o valor de quem interage com o ecrã e permitir um desenvolvimento mais fácil de aplicações ricas, que propiciem uma interação eficaz por parte de um ou mais utilizadores. 

Aliando o crescente uso de dispositivos móveis, como \textit{smartphones} e \textit{tablets}, aos recentes avanços tecnológicos existentes nos ecrãs públicos digitais, este projeto apresentava como principais objetivos a criação de uma \textit{framework} que facilite o desenvolvimento de aplicações de cariz público e a implementação de alguns exemplos destas aplicações, que permitam a utilização da \textit{framework} desenvolvida.

A \textit{framework} desenvolvida encontra-se orientada a objetos, sendo composta por 4 classes distintas. As subclasses representam os tipos de controlos que o programador terá disponíveis e poderá implementar na sua aplicação. Neste caso, foram desenvolvidos 3 diferentes tipos de widgets, que permitirão ao utilizador final interagir com a aplicação.

Como exemplo de aplicação foi implementado o clássico jogo da \textit{Snake}. Uma vez que não havia necessidade de desenvolver o jogo de raíz, foi necessário pesquisar por um exemplo em \textit{html} e \textit{javascript} ao qual se pudessem adaptar os controlos da \textit{framework}. Apesar de existirem um grande número de opções, foi escolhido o jogo referido, uma vez que é um jogo bastante conhecido, que não necessita de grandes explicações e num modo multi-jogador torna-se competitivo.

Não só no final, mas também ao longo do desenvolvimento, ocorreu a realização de alguns testes. Os testes realizados ao longo do desenvolvimento permitiram saber se a funcionalidade implementada fazia realmente o que é desejado, obtendo de forma rápida \textit{feedback} do trabalho acabado de realizar, prevenindo erros futuros semelhantes e trabalho desnecessário. 

No final foi pedido a 3 estudantes do MIEIC que integrassem a \textit{framework} desenvolvida com soluções \textit{open source} recorrendo à mesma para a definição dos respetivos controlos.


\chapter*{Abstract}

The presence of digital public displays in urban landscapes has increased, however only few of them allow a direct-manipulation to passersby. Nowadays, the main feature is to advertise a service or a product and often people ignore them. 

It is important to be centered in the final user and allow to developers an easier development of applications providing an efficient interaction for one or more users. 

Combining the growing use of mobile devices, such as smartphones and tablets, with the latest technological advances in existing public digital displays, this project had as main goals the creation of a framework that eases the development of public applications and the implementation of some examples allowing the use of the developed framework.

The developed object-oriented framework heavily relies in inheritance consisting of four distinct classes, one of them being an interface, that is the abstract representation of a control. The subclasses represent the types of controls that the developer will have available and can implement in an application. In this case, three different types of widgets, that allow the final user to interact with the application, have been developed.

As an example of application, it was implemented the classic game of Snake. Since there was no need to create the game, it was necessary to search for an example in HTML and JavaScript which it could apply the controls of framework. Although there were a large number of options, this game was chosen because it is well-known and, henceforth it does not need much explanation. Besides, in multiplayer mode the game becomes competitive.

Not only at the end, but also throughout development, some tests occurred. Tests during the development allowed whether the implemented functionality was the desired or not, quickly getting feedback from the most recent changes on the work, preventing similar future errors and unnecessary work.

At the end, three students from MIEIC were asked to incorporate the framework on open-source solutions using it for the respective controls implementation.