%!TEX root = mieic.tex
\chapter{Introdução} \label{chap:intro}

\section*{}
 
Esta dissertação, tem como tema \textit{Remote, direct-manipulation interaction for multi-user, web-based public display applications} e foi proposta pelo CITAR\footnote{Centro de Investigação em Ciência e Tecnologia das Artes, http://artes.ucp.pt/citar/} da Universidade Católica.  

\section{Contexto/Enquadramento} \label{sec:context}

Na atualidade, é cada vez maior o número de ecrãs públicos existentes em diversos cenários urbanos, sejam eles paragens de transportes públicos, salas de espera ou outras zonas mais movimentadas. No entanto, a maioria destes apenas é utilizada como meio de divulgação de determinado produto ou serviço, não permitindo ao transeunte interagir com o mesmo. A população já está habituada à sua presença, classificando-os muitas vezes como objetos inúteis, podendo facilmente passar despercebidos. 

Este cenário pode ser alterado, pois os recentes avanços da tecnologia podem proporcionar aos utilizadores interação com estes ecrãs através da manipulação direta dos mesmos, usando para isso o seu dispositivo móvel.

Apesar de já existir algum desenvolvimento nesta área, alterar o estado atual dos ecrãs em algo completamente novo requer algum investimento tecnológico e inovação.

Uma pesquisa mais cuidada revela que já começa a ser comum, em áreas mais movimentadas, como estações de comboios ou praças públicas a possibilidade de as pessoas interagirem com diversos ecrãs. Existem diferentes maneiras pelas quais esta interação é possível, como por exemplo através do toque, leitura de \textit{QR code}, introdução de um código, ou usando um \textit{kinnect} inserido no próprio ecrã. 

Nigel Davies~\cite{Davies2012b}, em \textit{Open Display Networks: A Communications Medium for the 21st Century}, compara a situação dos ecrãs públicos com os \textit{smartphones}, afirmando que a inovação ocorre quando são criados sistemas livres que encorajam a inovação e incentivam a um maior desenvolvimento de novos produtos.

\section{Projeto e Objetivos} \label{sec:proj}

O tema proposto tem como objetivo principal desenvolver e validar uma arquitetura que permita uma interação baseada no paradigma da manipulação directa, por outras palavras, tal como o nome indica representa uma interação que ocorre de forma direta, em que o utilizador se apercebe das alterações no exato momento.  

No final pretende-se obter uma \textit{framework} que facilite a criação de aplicações para ecrãs públicos, baseadas no paradigma acima descrito, bem como algumas aplicações exemplos que sejam construídas com base na API desenvolvida.

\section{Desafios} \label{sec:goals}

Como já foi referido anteriormente cada vez mais, em locais públicos existem diversos ecrãs, contudo a sua maioria apenas serve para publicitar determinado produto ou serviço. 

Esta é uma área nova, em constante desenvolvimento, em que há a possibilidade de estabelecer alguns conceitos de referência para o futuro.

Os ecrãs públicos estão situados em zonas estratégicas, encontrando-se maioritariamente localizados em zonas onde existe uma maior concentração de pessoas, sendo também importante estudar a possibilidade de uma interação por parte de mais do que  em vez de individual.

O presente projeto apresenta um lado desafiante que leva à procura de soluções para o desenvolvimento de aplicações web, que suportem uma interação por múltiplos utilizadores, usando para isso o seu próprio dispositivo móvel, levando também a pensar nos tipos de controlo que deverão estar disponíveis. 

Existem, para além dos desafios acima mencionados, outros aos quais poderá ser possível dar uma resposta, será, por exemplo, importante perceber de que modo a infra-estrutura da rede pode influenciar o correto funcionamento das aplicações, bem como se no mesmo ecrã poderá existir mais do que uma aplicação ou ainda se para cada aplicação existe a necessidade de haver um servidor diferente. 

Este trabalho irá também permitir o desenvolvimento de aplicações interativas mais ricas, abrindo o leque de possibilidades de configuração e interação com este tipo de aplicações. 

\section{Estrutura da Dissertação} \label{sec:struct}

Esta dissertação apresenta para além da introdução mais seis capítulos.

No capítulo~\ref{chap:biblio}, é descrito o estado da arte e são
apresentados trabalhos relacionados. 

No capítulo~\ref{chap:metod}, é apresentada a metodologia a usar no desenvolvimento do projeto. Segue-se o capítulo~\ref{chap:sol}, onde se descreve pormenorizadamente a solução implementada e as tecnologias usadas para alcançar os objetivos propostos.

No capítulo~\ref{chap:testes} são descritos os testes realizados e as conclusões tiradas a partir dos mesmos, sendo realizada uma discussão e análise crítica do produto desenvolvido no capítulo~\ref{chap:disc}.

Por último, capítulo~\ref{chap:concl}, o relatório termina com uma análise conclusiva e possíveis melhoramentos futuros.
