%!TEX root = mieic.tex
\chapter{Introdução} \label{chap:intro}

\section*{}
 
Esta dissertação, tem como tema \textit{Remote, direct-manipulation interaction for multi-user, web-based public display applications} e foi proposta pelo CITAR\footnote{Centro de Investigação em Ciência e Tecnologia das Artes} da Universidade Católica.  

\section{Contexto/Enquadramento} \label{sec:context}

Na atualidade, é cada vez maior o número de ecrãs públicos existentes em diversos cenários urbanos, sejam eles paragens de transportes públicos, salas de espera ou outras zonas mais movimentadas. No entanto, a maioria destes apenas é utilizada como meio de divulgação de determinado produto ou serviço, não permitindo ao transeunte interagir com o mesmo. A população já está habituada à sua presença, classificando-os muitas vezes como objetos inúteis, podendo facilmente passar despercebidos. 

Este cenário pode ser alterado, pois os recentes avanços da tecnologia podem proporcionar aos utilizadores interação com estes ecrãs através da manipulação direta dos mesmos, usando para isso o seu dispositivo móvel. 

Apesar de já existir algum desenvolvimento nesta área, alterar o estado atual dos ecrãs em algo completamente novo requer algum investimento tecnológico e inovação.

\section{Projeto e Objetivos} \label{sec:proj}

O tema proposto tem como objetivo principal desenvolver e validar uma arquitetura que permita uma interação baseada no paradigma direct-manipulation, pois poucos são os ecrãs existentes que possibilitam a manipulação directa através de um dispositivo móvel.

Inicialmente será realizado um estudo que permita ficar a conhecer de que forma a interação acima descrita pode ser inserida num novo toolkit, facilitando o desenvolvimento de aplicações para ecrãs públicos que requeiram o mesmo tipo de interação. 
O produto final será o toolkit  e algumas aplicações demo que demonstrem a versatilidade/relevância do mesmo, estas aplicações servirão no fundo para avaliar o trabalho desenvolvido e serão testadas nos ecrãs públicos que existem no CITAR.

Uma vez que o projeto apresenta um cariz público será importante ver respondidas algumas questões como:
\begin{itemize}
\item Como nos "ligamos" a um determinado ecrã? 
\item Que tecnologia/arquitetura de comunicação? 
\item Dado que a interação é pública, quem controla a aplicação num dado momento? 
\item Que tipo de eventos de alto-nível deve uma aplicação para ecrãs públicos receber?
\end{itemize}

\section{Desafios} \label{sec:goals}

Como já foi referido anteriormente cada vez mais, em locais públicos existem diversos ecrãs, contudo a sua maioria apenas serve para publicitar determinado produto ou serviço. 

Uma pesquisa mais cuidada revela que já começa a ser comum, em áreas mais movimentadas, como estações de comboios ou praças públicas a possibilidade de as pessoas interagirem com diversos ecrãs. Existem diferentes maneiras pelas quais esta interação é possível, como por exemplo através do toque, leitura de “QR code”, introdução de um código, ou usando um kinnect inserido no próprio ecrã, um dos objetivos passa também por definir a melhor maneira de o utilizador se ligar a detrminado ecrã.

Esta é uma área nova, em constante desenvolvimento, em que há a possibilidade de estabelecer alguns conceitos de referência para o futuro.

Os ecrãs públicos estão situados em zonas estratégicas, encontrando-se maioritariamente localizados em zonas onde existe uma maior concentração de pessoas, sendo também importante estudar a possibilidade de uma interação \textit{multi-user} em vez de individual.

Este trabalho irá também permitir o desenvolvimento de aplicações interativas mais ricas, abrindo o leque de possibilidades de configuração e interação com este tipo de aplicações. 

\section{Estrutura da Dissertação} \label{sec:struct}

Este relatório apresenta para além da introdução mais dois capítulos.

No capítulo~\ref{chap:biblio}, é descrito o estado da arte e são
apresentados trabalhos relacionados. 

No capítulo~\ref{chap:metod}, é apresentada a metodologia a usar no desenvolvimento do projeto.

No capítulo~\ref{chap:concl}, o relatório é concluído com a perspetiva da solução e planeamento do trabalho futuro.
